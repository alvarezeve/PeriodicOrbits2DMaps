
\chapter{Introducción}

El estudio de órbitas perri\'odicas hiperb\'olicas es un tema interesante en el estudio de los sistemas din\'amicos discretos. En particular el estudio de los conjuntos invariantes asociados a tales \'orbitas es un tema de intere\'es para quienes estudian la teor\'ia KAM (escribir completo el nombre). Uno de los problemas abiertos interesantes en sistemas din\'amicos es el estudio del rompimiento de toros invariantes para mapeos simpl\'ecticos. Tal problema puede ser abordado desde el estudio del comportamiento de las variedades estables e inestables de \'orbitas hiperb\'olicas. \\

En este trabajo se presenta la implementaci\'on de un m\'etodo para encontrar \'orbitas peri\'odicas de mapeos simpl\'ecticos de dos dimensiones y para parametrizar las variedades asociadas a aquellas \'orbitas que resulten ser hiperb\'olicas. El objetivo es dar una introducci\'on a la forma en la que se  estudia el rompimiento de los toros invariantes en este tipo de mapeos. 

