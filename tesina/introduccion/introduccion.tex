
\chapter{Introducción}

El estudio de órbitas peri\'odicas hiperb\'olicas es un tema interesante en el estudio de los sistemas din\'amicos discretos. En particular el estudio de los conjuntos invariantes asociados a tales \'orbitas es un tema de inter\'es para quienes estudian la teor\'ia KAM (Kolmogorov–Arnold–Moser). Uno de los problemas abiertos en sistemas din\'amicos es el estudio del rompimiento de toros invariantes para mapeos simpl\'ecticos. Tal problema puede ser abordado desde el estudio del comportamiento de las variedades estables e inestables de \'orbitas hiperb\'olicas. \\

En este trabajo se presenta la implementaci\'on de un m\'etodo para encontrar \'orbitas peri\'odicas de mapeos simpl\'ecticos de dos dimensiones y para parametrizar las variedades asociadas a aquellas \'orbitas que resulten ser hiperb\'olicas. El objetivo es dar una herramienta para el estudio del rompimiento de los toros invariantes. 

