
%%%%%%%%%%%%%%%%%%%%%%%%%%%%%%%%%%%%%%%%%%%%%%%%%%%%%%%%%%%%%%%%%%%%%%%%%
%           Capítulo 2: MARCO TEÓRICO - REVISIÓN DE LITERATURA
%%%%%%%%%%%%%%%%%%%%%%%%%%%%%%%%%%%%%%%%%%%%%%%%%%%%%%%%%%%%%%%%%%%%%%%%%
\chapter{Implementaci\'on del m\'etodo}
En este cap\'itulo se describe brevemente c\'omo se implement\'o el m\'etodo para encontrar puntos fijos. Sobre el m\'etodo de la parametrizaci\'on implementado se puede revisar la referencia \cite{eve} en donde se describe tambi\'en c\'omo se aplic\'o al mapeo est\'andar. 

\section{Desarrollo explícito para el mapeo estándar}

El mapeo et\'andar es uno de los mapeos m\'as estudiados, esta dado por 

\begin{equation}
M_{\kappa}(x_{n},y_{n}) = 
\left(\begin{array}{c}
x_{n} +y_{n+1} \\
y_{n}-\kappa\sin(2\pi x_{n})/2\pi
\end{array}\right),
\label{mapeo estandar}
\end{equation}
con $\kappa$ un par\'ametro y $x\in [0,1]$. Escrito de esta forma el mapeo se descompone en el producto de dos involuciones dadas por
\begin{equation}
I_{0}\left( \begin{array}{c}
x_{n}\\
y_{n}
\end{array}\right) = 
\left(\begin{array}{c}
-x_{n}\\
y_{n}-\frac{\kappa}{2\pi}\sin(2\pi x_{n})
\end{array}
\right),
\label{involucionA}
\end{equation}
\begin{equation}
I_{1}\left( \begin{array}{c}
x_{n}\\
y_{n}
\end{array}\right) = 
\left(\begin{array}{c}
y_{n}-x_{n}\\
y_{n}
\end{array}
\right).
\label{involucionB}
\end{equation}

Los conjuntos invariantes asociados a las involuciones \eqref{involucionA}, \eqref{involucionB} son los siguientes
\begin{equation}
\mathbb{J}_{0} = \{ x \in [0,1] |x=0, x=1/2\},
\label{invariantejA}
\end{equation}

\begin{equation}
\mathbb{J}_{1} = \{ x \in [0,1] |x=\frac{y}{2}, x=\frac{y+1}{2}\}.
\label{invariantejB}
\end{equation}

Siguiendo el resultado del Corolario 1, sea $(x_{0},y_{0})\in \mathbb{J}_{0}$ si buscamos una \'orbita de periodo 2 tendremos que resolver 
\begin{equation*}
M_{\kappa}(x,y) = M_{\kappa}(0,y)=(1/2,y),
\end{equation*}
lo cual se reduce a que
\begin{equation*}
y=1/2.
\end{equation*}
Entonces afirmamos que el punto $(0,1/2)$ es un punto de periodo 2, cuya\'orbita encontrada es $\{(0,1/2), (1/2,1/2)\}$. Es importante recordar que la variable $x$ est\'a definida en el intervalo $[0,1]$ por lo que se debe tomar m\'odulo 1 en esa componente. \\

Una vez encontrada un \'orbita peri\'odica es necesario determinar la estabilidad de la misma. Usando la linearizaci\'on alrededor del punto fijo, en este caso un punto en la \'orbita se toma como punto fijo  del mapeo  $M^{n}_{k}(x_{0},y_{0})=(x_{0},y_{0})$. La estabilidad esta dada por los valores propios del jacobiano $DM^{n}_{\kappa}(x_{0},y_{0})$. Para calcular este jacobiano no hace falta hacer la composici\'on, solo hace falta usar la regla de la cadena y la \'orbita peri\'odica.
\begin{align*}
DM^{n}_{\kappa}(x_{0},y_{0}) &= D(M_{\kappa}(M_{\kappa}(\cdots M_{\kappa}(x_{0},y_{0}))))\\
& = DM_{\kappa}(M^{n-1}(x_{0},y_{0}))DM_{\kappa}(M^{n-2}_{\kappa}(x_{n-2},y_{n-1}))\cdots DM_{\kappa}(x_{0},y_{0})\\
& = DM_{\kappa}(x_{n-1},y_{n-1})DM_{\kappa}(x_{n-2},y_{n-2})\cdots DM_{\kappa}(x_{0},y_{0}).
\end{align*}
Entonces el jacobiano $DM^{n}_{\kappa}(x_{0},y_{0})$ se puede calcular evaluando el jacobiano del mapeo est\'andar en los puntos de la \'orbita y multiplicando las matrices. El jacobiano del mapeo est\'andar es
\begin{eqnarray}
M^{n}_{k}(x_{n},y_{n})=\begin{pmatrix}
1 & 1 \\
- \kappa\cos(2\pi x_{n})& 1\\ 
\end{pmatrix},
\label{jacobiano mapeo}
\end{eqnarray}
y la \'orbita se conoce si se conoce solo uno de los puntos. Usando el hecho de que el determinante del producto es el producto de los determinantes se tiene que 
\begin{equation}
\textrm{det}DM^{n}_{\kappa}(x_{0},y_{0}) = \left(\textrm{det} DM(x_{n-1},y_{n-1})\right) \cdots \left(\textrm{det} DM(x_{0},y_{0})\right).
\label{determinante jacobiano m}
\end{equation} 
Si los valores propios asociados al polinomio caracter\'istico de la ecuaci\'on \eqref{determinante jacobiano m} tienen m\'odulo mayor que uno y menor que uno entonces tenemos que $(x_{0},y_{0})$ es un punto hiperb\'olico lo que indica que la \'orbita encontrada es hiperb\'olica. \\

Una vez que se tiene calculada una \'orbita peri\'odica hiperb\'olica del mapeo se aplica el m\'etodo de la parametrizaci\'on para describir las variedades invariantes asociadas a tal \'orbita. \\



Para el m\'etodo de parametrizaci\'on escribimos las variables ($x,y$) como dos polinomios de variable real $t$
\begin{eqnarray}
x(t)=\sum_{n=0}^{\infty}a_{n}t^{n}  ,
\label{x}
\end{eqnarray}
\begin{eqnarray}
y(t)=\sum_{n=0}^{\infty}b_{n}t^{n},
\label{y}
\end{eqnarray}
d\'onde $a_{0},b_{0}$ representan los coeficientes n-\'esimos del polinomio y tal que $\mathcal{P}(t):=(x(t),y(t))$. En cuanto a la dinámica interna $g$, se usa la ecuación $g(t)=\lambda t$ \eqref{funciong1}. Sustituyendo las variables en su expanci\'on de Taylor en $M\circ\mathcal{P}=\mathcal{P}\circ g$ \eqref{Ecua de invariancia}  y usando que $M$ es el mapeo est\'andar \eqref{mapeo estandar} obtenemos
\begin{eqnarray}
M_{\kappa}(x,y) = \left[\begin{array}{c}
x(t) + y(t) -\frac{\kappa}{2\pi}\sin(2\pi x(t)) \\
y(t) - \frac{\kappa}{2\pi}\sin[2\pi x(t)]
\end{array}\right] =\left[ \begin{array}{c}
x(\lambda t) \\
y(\lambda t)
\end{array}\right], 
\label{sumas en mapeo}
\end{eqnarray}
que en forma explícita es
\begin{eqnarray}
\left[\begin{array}{c}
\sum_{n=0}^{\infty}a_{n}t^{n} + \sum_{n=0}^{\infty}b_{n}t^{n} -\frac{\kappa}{2\pi}\sin\left(2\pi \sum_{n=0}^{\infty}a_{n}t^{n}\right)\\
\sum_{n=0}^{\infty}b_{n}t^{n} - \frac{\kappa}{2\pi}\sin(\sum_{n=0}^{\infty}b_{n}t^{n})
\end{array}\right] =\left[ \begin{array}{c}
\sum_{n=0}^{\infty}a_{n}\lambda^{n}t^{n} \\
\sum_{n=0}^{\infty}b_{n}\lambda^{n}t^{n}
\end{array}\right].
\label{expandida}
\end{eqnarray}
al desarrollar las sumas y usar la serie de Taylor el seno se puede obtener una ecuaci\'on para comparar los t\'erminos de \'orden 0. 

\begin{align*}
	\left[
	\begin{array}{c}
		\sum_{n=0}^{\infty}a_{n}t^{n} + \sum_{n=0}^{\infty}b_{n}t^{n}-\frac{\kappa}{2\pi}\sum_{n=0}^{\infty}\frac{(-1)^{n}}{(2n+1)!}\left(2\pi \sum_{n=0}^{\infty}a_{n}t^{n}\right)^{2n+1}\\
		\sum_{n=0}^{\infty}b_{n}t^{n}-\frac{\kappa}{2\pi}\sum_{n=0}^{\infty}\frac{(-1)^{n}}{(2n+1)!}\left(2\pi\sum_{n=0}^{\infty}a_{n}t^{n}\right)^{2n+1}
	\end{array}	\right]
	&=\\
	\left[
	\begin{array}{c}
		\sum_{n=0}^{\infty}a_{n}\lambda^{n}t^{n} \\
		\sum_{n=0}^{\infty}b_{n}\lambda^{n}t^{n}
	\end{array}
	\right]
\end{align*}
Desarrollando las sumas y agrupando los t\'erminos de \'orden cero.
\begin{equation}
	\left[
	\begin{array}{c}
	
			a_{0}+b_{0}-\frac{\kappa}{2\pi}\left(2\pi a_{0}-\frac{2\pi}{3!} 	a_{0}^{3}+\frac{2\pi}{5!} a_{0}^{5}-\cdots
			\right)\\
			b_{0}-\frac{\kappa}{2\pi}\left(2\pi a_{0}-\frac{2\pi}{3!}a_{0}^{3}+\frac{2\pi}{5!}a_{0}^{5}-\cdots\right)
	\end{array}
	\right]=
	\left[
	\begin{array}{c}
		a_{0}\\
		b_{0}
	\end{array}
	\right]
\end{equation}
De donde se obtiene que $a_{0},b_{0}$ son cero. An\'alogamente se pueden agrupar los de primer orden y obtener un sistema que se resulve a partir de los coeficientes $a_{0}$ y $b_{0}$. De manera sucesiva se puede obtener ecuaciones de recurrencia para encontrar los coeficientes de orden $j$, los detalles de c\'omo son estos c\'alculos se pueden consultar en \cite{eve}.


\section{Implementación del método}
La implementaci\'on del m\'etodo consiste principalmente en aplicar un m\'etodo de Newton para encontrar ra\'ices. Para encontrar las \'orbitas peri\'odicas, dado el mapeo y los conjuntos invariantes de las simetr\'ias se puede buscar una \'orbita peri\'odica de periodo $n$ usando una semilla. En la siguiente tabla se exquematiza c\'omo se calcula, sin embargo se recomienda que para mayor detalle viste \url{https://github.com/alvarezeve/Tesis-Variedades-Estables-e-inestables/}.
En la liga se encuentra la documentaci\'on necesaria para poder usar el m\'etodo as\'i como algunos ejemplos. \\
\linebreak
\linebreak
\linebreak
\begin{center}
	\textcolor{blue}{\textbf{Pasos}}
	\begin{tabbing}
	12\=1234567890123456789012345678901234567890123456\=12345678901234567890123456\kill%
	\>............................................................  \>..................................................\\
	\>\textbf{1.}  Se reciben el mapeo, las involuciones,  \> \\
	\>el periodo y una semilla. \>$M, I_{A}, I_{B},n,v_{0}$ \\
	\>............................................................  \>..................................................\\
	\>\textbf{2.} Se crea un nuevo mapeo que es \> \\
	\> la composición del mapeo $M$, $k$ veces.\>  $f(v)=M^{k}(v)$ \\
	\>............................................................  \>..................................................\\
	\>\textbf{3.} Se toma un punto en el conjunto  \> $J_{A}$\\
	\> invariante. \> \\
	\>............................................................  \>..................................................\\
	\>\textbf{4.} Se usa la función creada en el paso 2. \> \\
	\> y se aplica al punto elegido en el paso \> $f(p)$\\
	\> anterior.\> \\
	\>............................................................  \>..................................................\\
	\>\textbf{5.} Se usa la ecuación que describe \> \\
	\> el conjunto $J_{B}$ y el hecho de que es  \> $g(f(p))=0$\\
	\>invariante para construir una nueva  \> \\
	\> ecuación.\\
	\>............................................................  \>..................................................\\
	\>\textbf{6.}  Se usa un método de Newton para \> \\
	\> encontrar la raíz de la ecuación del \> \\
	\> paso 5.\\
	\>............................................................  \>..................................................\\
	\end{tabbing} 
\end{center}

La implementaci\'on del m\'etodo de parametrizaci\'on fue resultado de un trabajo anterior \cite{eve}, por lo que solo se decribir\'a de manera muy superficial lo programado. Al igual que en el caso del m\'etodo para las \'orbitas peri\'odicas se puede encontrar la documentaci\'on en PONER URL \url{https://github.com/alvarezeve/Tesis-Variedades-Estables-e-inestables/}. 

\begin{center}
	\textcolor{blue}{\textbf{Pasos}}
	\begin{tabbing}
		12\=1234567890123456789012345678901234567890123456\=12345678901234567890123456\kill%
		\>............................................................  \>..................................................\\
		\>\textbf{1.} Se reciben el mapeo, el punto fijo  \> \\
		\> el orden del polinomio, el intervalo en  \> $M, O, v, t, \Delta t$\\
		\> el que se va a evaluar y el paso.\> \\
		\>............................................................  \>..................................................\\
		\>\textbf{2.} Se crean dos polinomios de grado 1  \> $P_{x} = a_{0}+a_{1}t+O(t^{2})$\\
		\> asociados a las variables del mapeo. \>  $P_{y} = b_{0} +b_{1}t+O(t^{2})$ \\
		\>............................................................  \>..................................................\\
		\>\textbf{3.} Se aplica el mapeo a los polinomios.  \> $M(P_{x}+P_{y})$\\
		\>............................................................  \>..................................................\\
		\>\textbf{4.} Se calcula la matriz jacobiana del  \> \\
		\>sistema anterior y se calculan los valores \> $JM(P_{x},P_{y})$\\
		\> y vectores propios.\\
		\>............................................................  \>..................................................\\
		\>\textbf{5.} Se elige el valor propio asociado a la  \> \\
		\>variedad que se está calculando \> $\lambda$\\
		\>............................................................  \>..................................................\\
		\>\textbf{6.}Se resuelve para $a_{1},b_{1}$.\> $a_{1}(a_{0}, \lambda), b_{1}(b_{0},\lambda)$\\
		\> \> \\
		\>............................................................  \>..................................................\\
		\> \textbf{7. }Se sustituyen los valores de $a_{1},b_{1}$ \> $P_{x} = a_{0}+a_{1}t+a_{2}t^{2}+O(t^{3})$\\
		\> en los polinomios $P_{x},P_{y}$ y se aumenta\>  $P_{y} = b_{0} +b_{1}t +b_{2}t^{2}+O(t^{3})$\\
		\>  el orden.\> \\
		\>............................................................  \>..................................................\\
		\>\textbf{8.} Se aplica el mapeo a los nuevos\>$M(P_{x}+P_{y})$ \\
		\> polinomios.\>\\
		\>............................................................  \>..................................................\\
		\>\textbf{9.} Se crea el polinomio \> $P_{\lambda x} = a_{0}+a_{1}\lambda t+a_{2}\lambda t^{2}+O(t^{3})$\\
		\> \>$P_{\lambda y} = b_{0}+b_{1}\lambda t+b_{2}\lambda t^{2}+O(t^{3})$\\
		\>............................................................  \>..................................................\\
		\>\textbf{10.} Se crea una ecuación con la resta \> $M(P_{x}+P_{y}) -(P_{\lambda x},P_{\lambda y})$=0 \\
		\> de las ecuaciones del paso 8 y 9.\> \\
		\>............................................................  \>..................................................\\
		\> \textbf{11.} Se resuelve la ecuación del paso 10 \>  \\
		\> para las variables $a_{2},b_{2}$. \>\\
		\>............................................................  \>..................................................\\
		\> \textbf{12.} Se regresa al paso 7. y se itera\> \\
		\> hasta que se llegue al orden deseado. \> \\
		\>............................................................  \>..................................................\\
		
		
		
		
	\end{tabbing} 
\end{center}


En ambas implementaciones se añadi\'o una forma de evaluar el error num\'erico. Para las \'orbitas peri\'odicas se tom\'o la diferencia entre el punto de la \'orbita de periodo $n$ encontrado y su n-\'esima iteraci\'on .
\begin{equation}
	\mathbb{E} = ||(x_{0},y_{0})-M^{n}(x_{0},y_{0})||
	\label{errororbitasperiodicas}
\end{equation}
Mientras que en el caso de las parametrizaciones el error se toma de la diferencia entre composiciones evaluada en el intervalo en el que se tom\'o la parametrizaci\'on. 
\begin{equation}
	\mathbf{E} = || M\circ \mathcal{P}- \mathcal{P}\circ g||
\end{equation}
Los errores se analizan por separado ya que cada m\'etodo puede considerarse ajeno. 






